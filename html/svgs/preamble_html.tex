\RequirePackage{etex}
% \usepackage[T1]{fontenc}
\usepackage{natbib}

\newcommand{\bookversion}{1.2.0\xspace}
\newcommand{\bookdate}{August 2025\xspace}

\newif\ifsolutions % when set to true, additional text appears containing solutions
\solutionsfalse
% \solutionstrue

\newif\ifcourse % To remove stuff that shouldn't be present for the preprint release
% \coursefalse
\coursetrue

\newif\ifblackwhite % When set to True, makes the book use black-and-white style, instead of colored ones.
\blackwhitefalse
% \blackwhitetrue

\newif\iftikzitdraft % Whether `draft' mode should be used for tikzit, which uses pre-built PDF files.
\tikzitdraftfalse
% \tikzitdrafttrue

\newcommand{\ismain}{0} % For chapters to distinguish whether they are being called as the main file, or as a subfile of main.tex

\renewcommand{\theenumi}{\arabic{enumi}.}
\setcounter{tocdepth}{2}
\setcounter{secnumdepth}{3}

% \usepackage{lmodern}

\usepackage{morewrites} % Prevent ``No room for a new \write'' error.

\usepackage{xr} % for having references to external documents

% \usepackage[activate=true,
%     final,
%     tracking=true,
%     % kerning=true,
%     % spacing=true,
%     factor=1100,
%     stretch=10,
%     shrink=10]{microtype}
% \microtypecontext{spacing=nonfrench}

\newcommand{\bw}{0}
\newcommand{\showoptional}{1}

\newcommand{\chapterquote}[2]{%
\begin{quote}
    \ifblackwhite
    \color{gray!50!black}
    \else
    \color{blue!50!black}
    \fi
  #1

  \medskip

  \hfill - \textit{#2}
\end{quote}

\bigskip
}

% \font\omding=omding

\usepackage{amsmath,amsthm,amssymb}
\usepackage{mathtools}
\usepackage{xspace,enumerate,epsfig}
\usepackage{graphicx}
\graphicspath{{.}{./figures/}}
\usepackage{marvosym}
\usepackage{bm}
\usepackage{tabu}
\tabulinesep=2mm

\usepackage{stmaryrd}
% \usepackage{docmute}
\usepackage{subfiles}
\usepackage{keycommand}

%% enables using PDF figures in cache/, if they exist
\iftikzitdraft
    \usepackage[draft]{tikzit}
    \else
    \usepackage{tikzit}
\fi
\usepackage{nicematrix}
% \usetikzlibrary{external}
% \tikzexternalize[prefix=tikz-cache/]

% THIS FILE IS AUTOMATICALLY GENERATED, DO NOT EDIT

\usepackage{pgfkeys}
\newcommand{\setcoderef}[3]{\pgfkeyssetvalue{/zxbook/codeexer/#1}{#3}}
\newcommand{\coderef}[1]{\pgfkeysifdefined{/zxbook/codeexer/#1}{\pgfkeys{/zxbook/codeexer/#1}}{??}}

\setcoderef{basic-states}{PQS_2_Circuits}{2.1}
\setcoderef{cnot-tonc}{PQS_2_Circuits}{2.2}
\setcoderef{z-phase}{PQS_2_Circuits}{2.3}
\setcoderef{euler-h-concrete}{PQS_2_Circuits}{2.4}
\setcoderef{parity-phase}{PQS_2_Circuits}{2.5}
\setcoderef{x-merge}{PQS_3_ZX}{3.1}
\setcoderef{tensor-compose}{PQS_3_ZX}{3.2}
\setcoderef{cnot-qasm}{PQS_3_ZX}{3.3}
\setcoderef{compare-tensors}{PQS_3_ZX}{3.4}
\setcoderef{zx-basis}{PQS_3_ZX}{3.5}
\setcoderef{zx-circuit-amplitudes}{PQS_3_ZX}{3.6}
\setcoderef{pyzx-three-cnot}{PQS_3_ZX}{3.7}
\setcoderef{zx-compose-adjoint}{PQS_3_ZX}{3.8}
\setcoderef{circuit-optimise}{PQS_3_ZX}{3.9}
\setcoderef{optimise-random-circuits}{PQS_3_ZX}{3.10}
\setcoderef{zx-simplify}{PQS_3_ZX}{3.11}
\setcoderef{zx-optimise}{PQS_3_ZX}{3.12}
\setcoderef{compare-zx-optimise}{PQS_3_ZX}{3.13}


\usepackage[hyphens]{url}
\usepackage{hyperref}

\ifblackwhite
    \hypersetup{
         colorlinks = true,
         linkcolor = gray!50!black,
         anchorcolor = gray!50!black,
         citecolor = gray!50!black,
         filecolor = gray!50!black,
         urlcolor = gray!50!black
    }
    \else % use colours
    \hypersetup{
         colorlinks = true,
         linkcolor = blue!70!black,
         anchorcolor = blue!70!black,
         citecolor = red!70!black,
         filecolor = blue!70!black,
         urlcolor = blue!70!black
    }
\fi
\usepackage[hyphenbreaks]{breakurl}
\usepackage[all]{hypcap}

\ifblackwhite
    \input{circuits-bw.tikzdefs}
    \input{circuits-bw.tikzstyles}
\else
    \input{circuits.tikzdefs}
    \input{circuits.tikzstyles}
\fi

\usepackage[ruled,nofillcomment]{algorithm2e}
\SetKwProg{Proc}{Procedure}{}{end}
\newcommand{\comm}[1]{\textit{\color{zxgreenfg} #1}}
\newcommand{\keyword}[1]{\textbf{\color{zxredfg} #1}}
\SetCommentSty{comm}
\SetKwSty{keyword}

% \usepackage{palatino}

\usepackage{ifthen}
% \usepackage{makeidx}
% \makeindex

\newcommand{\dfnoindex}[1]{\textbf{#1}}

% \newcommand{\dfnoindex}[1]{\textbf{\color{zxredfg} #1}}
\newkeycommand{\df}[index=][1]{%
% \ifthenelse{\equal{\commandkey{index}}{}}
    % {\index{#1}}
    % {\index{\commandkey{index}}}%
\dfnoindex{#1}}


% OURS
\theoremstyle{definition}
\newtheorem{theorem}{Theorem}[section]
\newtheorem{corollary}[theorem]{Corollary}
\newtheorem{lemma}[theorem]{Lemma}
\newtheorem{proposition}[theorem]{Proposition}
\newtheorem{conjecture}[theorem]{Conjecture}
\newtheorem{definition}[theorem]{Definition}
\newtheorem{problem}[theorem]{Problem}
\newtheorem{fact}[theorem]{Fact}
\newtheorem{example}[theorem]{Example}
\newtheorem{examples}[theorem]{Examples}
\newtheorem{example*}[theorem]{Example*}
\newtheorem{examples*}[theorem]{Examples*}
\newtheorem{remark}[theorem]{Remark}
\newtheorem{remark*}[theorem]{Remark*}
\newtheorem{question}[theorem]{Question}
\newtheorem{assumption}[theorem]{Assumption}
\newtheorem{algo}[theorem]{Algorithm}

\newtheorem{jupyter}[theorem]{Coding exercise}

%\newtheoremstyle{exercise}{3pt}{3pt}{\color{red}}{}{\bf}{}{.5em}{}
%\theoremstyle{exercise}
\newtheorem{exer}{Exercise}[chapter]
\newtheorem{exer*}[exer]{Exercise*}

\usepackage{comment}

\newenvironment{solution}{%
    \mbox{}\newline
    \textbf{Solution:} \dotfill\newline
    }
    {
    \ifsolutions
    \mbox{}\newline
    \textbf{End Solution}\dotfill
    \fi
}

\unless\ifsolutions
\excludecomment{solution}
\fi


\newcommand{\TODOa}[1]{{\color{blue} [\textbf{TODO:} #1]}}
\newcommand{\TODOj}[1]{{\color{green!50!black} [\textbf{TODO:} #1]}}
\newcommand{\NOTEa}[1]{{\color{blue} [\textbf{a:} #1]}}
\newcommand{\NOTEj}[1]{{\color{green!50!black} [\textbf{j:} #1]}}

% HIDE TODOs and NOTEs
\ifcourse
\renewcommand{\TODOa}[1]{}
\renewcommand{\TODOj}[1]{}
\renewcommand{\NOTEa}[1]{}
\renewcommand{\NOTEj}[1]{}
\fi

\newcommand{\CHECK}[1]{\marginpar{\scriptsize\bR \textbf{CHECK:} #1\e}}

\newcommand{\breakrule}{{\bigskip\huge\bR$\%\%\%\%\%\%\%\%\%\%\%\%\%\%\%\%\%\%\%\%\%\%\%\%\%$\e\bigskip}}

\newcommand{\hint}[1]{\noindent\textit{Hint: #1}}


\hyphenation{line-break line-breaks docu-ment triangle cambridge amsthdoc
  cambridgemods baseline-skip author authors cambridgestyle en-vir-on-ment polar
  ortho-normal}




% \usepackage[color,leftbars]{changebar}
\usepackage[color]{changebar}

\def\cbB{\cbcolor{blue}\cbstart}
\def\cbR{\cbcolor{red}\cbstart}

%begin Bob's
\usepackage{color}
\def\bR{\begin{color}{red}} 
\def\bB{\begin{color}{blue}}
\def\bM{\begin{color}{magenta}}
\def\bC{\begin{color}{cyan}}
\def\bW{\begin{color}{white}}
\def\bBl{\begin{color}{black}} 
\def\bG{\begin{color}{green}}
\def\bY{\begin{color}{yellow}}
\def\e{\end{color}\xspace}
\newcommand{\bit}{\begin{itemize}}
\newcommand{\eit}{\end{itemize}\par\noindent}
\newcommand{\ben}{\begin{enumerate}}
\newcommand{\een}{\end{enumerate}\par\noindent}
\newcommand{\beq}{\begin{equation}}
\newcommand{\eeq}{\end{equation}\par\noindent}
\newcommand{\beqa}{\begin{eqnarray*}}
\newcommand{\eeqa}{\end{eqnarray*}\par\noindent}
\newcommand{\beqn}{\begin{eqnarray}}
\newcommand{\eeqn}{\end{eqnarray}\par\noindent}
%end Bob's

% hide certain colours

% \def\bR{\begin{color}{black}} 
% \def\bB{\begin{color}{black}}
% \def\bM{\begin{color}{black}}
% \def\bC{\begin{color}{black}}
% \def\bW{\begin{color}{black}}
% \def\bG{\begin{color}{black}}
% \def\bY{\begin{color}{black}}



%\NiceMatrixOptions{code-for-last-col = \color{highlight_text}\small}
\newcommand{\bitpauli}[3]{\big(#1\,|\,#2\big)\,\textrm{\color{highlight_text}\small$\bm{#3}$ } }
\DeclarePairedDelimiter{\inn}{\langle}{\rangle}

%\newcommand{\lp}{\bm{0}}
%\newcommand{\lm}{\bm{2}}
%\newcommand{\lpi}{\bm{1}}
%\newcommand{\lmi}{\bm{3}}

\newcommand{\lp}{\texttt{\color{highlight_text}\small\textbf{+}}}
\newcommand{\lm}{\texttt{\color{highlight_text}\small\textbf{-}}}
\newcommand{\lpi}{\texttt{\color{highlight_text}\small\textbf{+$\bm{i}$}}}
\newcommand{\lmi}{\texttt{\color{highlight_text}\small\textbf{-$\bm{i}$}}}


\newcommand{\biggline}{\bigg|}
% \DeclareMathOperator\Dom{Dom}
% \DeclareMathOperator\Zer{Zer}
% \DeclareMathOperator\Dim{Dim}
\DeclareMathOperator\diag{diag}
% \DeclareMathOperator\Re{Re}
% \DeclareMathOperator\Im{Im}

\setcounter{MaxMatrixCols}{20} % Needed if you want pmatrix with more than 10 columns
